\newcommand{\pienst}{\Pi_{\Omega}}
\newcommand{\penst}{P_{\Omega}}
\newcommand{\aenst}{A_{\Omega}}
\newcommand{\benst}{B_{\Omega}}
\newcommand{\denst}{D_{\Omega}}
\newcommand{\matderv}[1]{\frac{D#1}{Dt}}

\subsection{Term-by-Term Enstrophy Transport Evaluation}
From the enstrophy transport equation
\begin{equation}
    \underbrace{\pdv{\Omega}{t} + \widetilde{\mathbf{u}} \cdot \grad{\Omega}}_{\equiv \matderv{\Omega}} =
        \underbrace{\widetilde{\mathbf{\omega}} \cdot \widetilde{\mathbf{S}} \cdot \widetilde{\mathbf{\omega}}}_{\equiv A_{\Omega}}
        \underbrace{+ \nu \div \left(\grad{\Omega}\right)}_{\equiv B_{\Omega}}
        \underbrace{- \nu \grad{\widetilde{\omega}} : \grad{\widetilde{\omega}}}_{\equiv D_{\Omega}}
        \underbrace{- \div\left(\widetilde{\mathbf{\omega}} \mathbf{\Psi}\right)}_{\equiv \Pi_{\Omega}} 
        \underbrace{+ \mathbf{\Psi} \grad{\widetilde{\omega}}}_{\equiv P_{\Omega}} 
        \label{eq:enstrophy-transport}
\end{equation}
we can define the following budget terms


\begin{subequations}
    \begin{align}
        A_{\Omega} \equiv &
            \widetilde{\mathbf{\omega}} 
            \cdot \widetilde{\mathbf{S}} 
            \cdot \widetilde{\mathbf{\omega}}   \\
        B_{\Omega} \equiv &
            \nu \div{\grad{\Omega}} \\
        D_{\Omega} \equiv &
            - \nu \grad{\widetilde{\omega}} : \grad{\widetilde{\omega}} \\
        \Pi_{\Omega} \equiv &
            -\div\left(\widetilde{\mathbf{\omega}} \mathbf{\Psi}\right) \\
        P_{\Omega} \equiv &
            \mathbf{\Psi} \grad{\widetilde{\omega}}
    \end{align}
\end{subequations}

On the right side of Eq.~\ref{eq:enstrophy-transport}
\begin{enumerate}
    \item   
        Term $A_{\Omega}$ accounts for the enstrophy production due to the vortex stretching in the
        resolved field. Since the vorticity tries align itself  with the most extensional strain
        rate component, we expect this term to largely positive and thus adding to the flows total
        enstrophy.
        
    \item
        Term $\benst$ accounts for the redistribution of enstrophy within in the resolved
        scales. Since $\benst$ can be written in ``flux form'' $\div{(\;)}$, meaning it appears
        as a divergence of the quantity in parenthesis, and thus cannot correspond to the net
        addition or removal of enstrophy $\Omega$ from the flow as a whole. Instead, it only acts
        to redistribute the enstrophy within the flow; any increase in $\Omega$ that it produces at
        one location must be offset by a corresponding decrease in $\Omega$ at another point.   

    \item
        Term $\denst$ accounts for the resolved-scale dissipation of enstrophy. Since 
        $\grad{\widetilde{\omega}}: \grad{\widetilde{\omega}}$ is a square term, $\denst$ is always
        non-positive, thus everywhere in the flow it always acts to remove enstrophy form the
        resolved scales.
       
    \item
        Term $\pienst$ accounts for the redistribution of the enstrophy with in the resolved scales
        by the subgrid stress; it analogous to term $\benst$ but accounts for the redistribution by
        the subgrid stresses rather than by the viscous stresses.

    \item
        Term $\penst$ accounts for the subgrid production of enstrophy, namely the enstrophy
        exchange between the resolved and subgrid scales. This term can be locally positive or
        negative; whenever $\penst < 0 $ there is enstrophy transfer from the resolved scales into
        the subgrid scales (''forward scatter'' of enstrophy), and whenever $\penst >0$ there is an
        enstrophy transfer from the subgrid scales into the resolved scales (``backscatter'' of
        enstrophy).
        
\end{enumerate}
