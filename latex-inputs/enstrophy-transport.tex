\newcommand{\pienst}{\Pi_{\Omega}}
\newcommand{\penst}{P_{\Omega}}
\newcommand{\aenst}{A_{\Omega}}
\newcommand{\benst}{B_{\Omega}}
\newcommand{\denst}{D_{\Omega}}
\newcommand{\matderv}[1]{\frac{D#1}{Dt}}

\subsection{Term-by-Term Enstrophy Transport Evaluation}
To understand the blowup and healing process of the energetics it makes sense to examine the
transport equation for the filtered enstrophy. We start by substituting in
Eqs.~\ref{eq:SGS-enstrophy-production} \& \ref{eq:SGS-enstrophy-transport} into
Eq.~\ref{eq:enstrophy-vector} to get the following in index notation
\begin{equation}
    \underbrace{\pdv{\Omega}{t} + \widetilde{u}_{j} \pdv{\Omega}{x_{i}}}_{\equiv \matderv{\Omega}} =
        \underbrace{\widetilde{\omega}_{i} \widetilde{S}_{ij} \widetilde{\omega}_{j}}_{\equiv A_{\Omega}}
        \underbrace{+ \nu \pdv{}{x_{j}} \left( \pdv{\Omega}{x_{j}} \right)}_{\equiv B_{\Omega}}
        \underbrace{- \nu \pdv{\widetilde{\omega_{i}}}{x_{j}}\pdv{\widetilde{\omega}_{i}}{x_{j}}}_{\equiv D_{\Omega}}
        \underbrace{- \pdv{}{x_{j}} \left(\widetilde{\omega} \Psi_{il}\right)}_{\equiv \Pi_{\Omega}} 
        \underbrace{+ \Psi_{il} \pdv{\widetilde{\omega}_{i}}{x_{j}}}_{\equiv P_{\Omega}} 
        \label{eq:enstrophy-transport}
\end{equation}
where we can define the following budget terms as
\begin{subequations}
    \begin{align}
        A_{\Omega} \equiv \; &
            \widetilde{\omega}_{i} \widetilde{S}_{ij} \widetilde{\omega}_{j}\\
        B_{\Omega} \equiv \; &
            \nu \pdv{}{x_{j}} \left( \pdv{\Omega}{x_{j}} \right)\\
        D_{\Omega} \equiv \; &
            -\nu \pdv{\widetilde{\omega_{i}}}{x_{j}}\pdv{\widetilde{\omega}_{i}}{x_{j}}\\
        \Pi_{\Omega} \equiv \; &
            -\pdv{}{x_{j}} \left(\widetilde{\omega} \Psi_{il}\right)\\
        P_{\Omega} \equiv \; &
            \Psi_{il} \pdv{\widetilde{\omega}_{i}}{x_{j}}
    \end{align}
\end{subequations}

On the right side of Eq.~\ref{eq:enstrophy-transport}
\begin{enumerate}
    \item   
        Term $A_{\Omega}$ accounts for the enstrophy production due to the vortex stretching in the
        resolved field. Since the vorticity tries align itself  with the most extensional strain
        rate component, we expect this term to largely positive and thus adding to the flows total
        enstrophy.
        
    \item
        Term $\benst$ accounts for the redistribution of enstrophy within in the resolved scales.
        Since $\benst$ can be written in ``flux form'' $\div \equiv \pdv*{(\;)}{x_{j}}$, meaning it
        appears as a divergence of the quantity in parenthesis, and thus cannot correspond to the
        net addition or removal of enstrophy $\Omega$ from the flow as a whole. Instead, it only
        acts to redistribute the enstrophy within the flow; any increase in $\Omega$ that it
        produces at one location must be offset by a corresponding decrease in $\Omega$ at another
        point.   

    \item
        Term $\denst$ accounts for the resolved-scale dissipation of enstrophy. Since
        $\left(\pdv*{\widetilde{\omega}_{i}}{x_{j}} \pdv*{\widetilde{\omega}_{i}}{x_{j}}\right)$ is
        a square term, $\denst$ is always non-positive, thus everywhere in the flow it always acts
        to remove enstrophy form the resolved scales.
       
    \item
        Term $\pienst$ accounts for the redistribution of the enstrophy with in the resolved scales
        by the subgrid stress; it analogous to term $\benst$ but accounts for the redistribution by
        the subgrid stresses rather than by the viscous stresses.

    \item
        Term $\penst$ accounts for the subgrid production of enstrophy, namely the enstrophy
        exchange between the resolved and subgrid scales. This term can be locally positive or
        negative; whenever $\penst < 0 $ there is enstrophy transfer from the resolved scales into
        the subgrid scales (''forward scatter'' of enstrophy), and whenever $\penst >0$ there is an
        enstrophy transfer from the subgrid scales into the resolved scales (``backscatter'' of
        enstrophy).
        
\end{enumerate}
