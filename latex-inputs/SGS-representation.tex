\newcommand{\sgsvec}{\mathbf{\tau}} 
\newcommand{\svec}{\mathbf{S}}
\newcommand{\rvec}{\mathbf{R}}
%\newpage
\section{Kinetic energy transport}
Staring with the filtered momentum transport equation

\begin{subequations}
    \begin{equation}
        \pdv{\widetilde{u}_{i}}{t} + \widetilde{u}_{j} \pdv{\widetilde{u}_{i}}{x_{j}} = 
            -\frac{1}{\rho} \pdv{\widetilde{p}}{x_{i}} 
            + \nu \pdv{\widetilde{u}_{i}}{x_{j}}{x_{j}}
            - \pdv{}{x_{j}}\big( \underbrace{\widetilde{u_{i} u_{j}} - \widetilde{u}_{i} \widetilde{u}_{j}}_{\equiv \tau_{ij}} \big)
    \end{equation}
    \begin{equation}
        \pdv{\widetilde{u}_{i}}{t} + \widetilde{u}_{j} \pdv{\widetilde{u}_{i}}{x_{j}} = 
            -\frac{1}{\rho} \pdv{\widetilde{p}}{x_{i}} 
            + \nu \pdv{\widetilde{u}_{i}}{x_{j}}{x_{j}}
            - \pdv{\tau_{ij}}{x_{j}}
    \end{equation}
    \label{eq:ns-filt}
\end{subequations}

The kinetic energy $k\equiv \frac{1}{2} \widetilde{u}_{i} \widetilde{u}_{i}$
can be easily derived by multiplying Eq.~(\ref{eq:ns-filt}) by $\widetilde{u}_{i}$
\begin{equation}
    \widetilde{u}_{i} \bigg(
    \pdv{\widetilde{u}_{i}}{t} + \widetilde{u}_{j} \pdv{\widetilde{u}_{i}}{x_{j}} = 
        -\frac{1}{\rho} \pdv{\widetilde{p}}{x_{i}} 
        + \nu \pdv{\widetilde{u}_{i}}{x_{j}}{x_{j}}
        - \pdv{\tau_{ij}}{x_{j}}
        \bigg)
\end{equation}
and evaluating each term by term we get the following
\begin{subequations}
    \begin{equation}
        \widetilde{u}_{i} \pdv{\widetilde{u}_{i}}{t} =
            \pdv{}{t} \bigg(\frac{1}{2}\widetilde{u}_{i} \widetilde{u}_{i}\bigg) =
            \pdv{k}{t}
    \end{equation}
    \begin{equation}
        \widetilde{u}_{i} \widetilde{u}_{j} \pdv{\widetilde{u}_{i}}{x_{j}}  =
        \widetilde{u}_{j} \pdv{}{x_{j}} \bigg(\frac{1}{2} \widetilde{u}_{i}\widetilde{u}_{i} \bigg) = 
        \widetilde{u}_{j} \pdv{k}{x_{j}}
    \end{equation}
    \begin{equation}
         - \frac{1}{\rho} \widetilde{u}_{i} \pdv{\widetilde{p}}{x_{i}} = 
           -\frac{1}{\rho} \pdv{}{x_{i}} \bigg(\widetilde{p} \widetilde{u}_{i}\bigg) 
    \end{equation}
    \begin{equation}
        \nu \widetilde{u}_{i}  \pdv{}{x_{j}} \left( \pdv{\widetilde{u}_{i}}{x_{j}} \right) = 
            \nu \Bigg[ \pdv{}{x_{j}} \bigg( \widetilde{u}_{i} \pdv{\widetilde{u}_{i}}{x_{j}}\bigg) -
                \pdv {\widetilde{u}_{i}}{x_{j}} \pdv {\widetilde{u}_{i}}{x_{j}} \bigg] = 
            2\nu \Bigg[ \pdv{}{x_{j}} \bigg( \widetilde{u}_{i} \widetilde{S}_{ij}\bigg) -
                \widetilde{S}_{ij} \widetilde{S}_{ij} \bigg]
    \end{equation}
    \begin{equation}
        -\widetilde{u}_{i} \pdv{\tau_{ij}}{x_{j}} = 
          - \pdv{}{x_{j}} \left(\widetilde{u}_{i} \tau_{ij}\right) + \tau_{ij} \pdv{\widetilde{u}_{i}}{x_{j}} =
          - \pdv{}{x_{j}} \left(\widetilde{u}_{i} \tau_{ij}\right) + \tau_{ij} \widetilde{S}_{ij} 
    \end{equation}
\end{subequations}

Thus we can express the transport of kinetic energy in index notation as 

\begin{equation}
    \colorboxed{red}{
        \underbrace{ \pdv{k}{t} + \widetilde{u}_{j} \pdv{\widetilde{k}}{x_{j}} }_{\frac{Dk}{Dt}} =
            \underbrace{ - \frac{1}{\rho} \pdv{}{x_{j}} \bigg( \widetilde{p}\widetilde{u}_{j}\bigg) }_{\equiv A}
            \underbrace{ + 2\nu \pdv{}{x_{j}}\bigg(\widetilde{u}_{i} \widetilde{S}_{ij} \bigg) }_{\equiv B} 
            \underbrace{ - 2\nu \widetilde{S}_{ij} \widetilde{S}_{ij} }_{\equiv D}
            \underbrace{ - \pdv{}{x_{j}}\bigg(\widetilde{u}_{i} \tau_{ij}\bigg) }_{\equiv C}
            \underbrace{ + \tau_{ij} \widetilde{S}_{ij}}_{\equiv P}
        }
\end{equation}
 
 and in vector notation as

 \newcommand{\vectil}[1]{\mathbf{\widetilde{#1}}}

\begin{equation}
    \colorboxed{red}{
        \underbrace{ \pdv{k}{t} + \vectil{u} \cdot \grad{k} }_{\frac{Dk}{Dt}} = 
            \underbrace{ - \frac{1}{\rho} \div \left(\widetilde{p} \vectil{u} \right) }_{\equiv A} 
            \underbrace{ + 2 \nu \div \left( \vectil{u} \cdot \vectil{S} \right) }_{\equiv B}
            \underbrace{ - 2 \nu \vectil{S}:\vectil{S} }_{\equiv D}
            \underbrace{ - \div \left( \vectil{u} \cdot \mathbf{\tau} \right) }_{\equiv C}
            \underbrace{ + \mathbf{\tau} : \vectil{S} }_{\equiv P}
        }
\end{equation}

where we can define the following budget terms as

\begin{subequations}




\subsection{SGS Representation}
From the ``board notes'' the kinetic energy production term in the kinetic
energy transport equation can be defined as
\begin{equation}
    P =  - \sgsvec:\svec
\end{equation}
Additionally, applying the following  formulation of the subgrid stress
(SGS) 
\begin{equation}
    \sgsvec = c_{1} \svec + c_{2} \svec^2 + c_{3} \rvec^2
                + c_{4} \left(\svec \rvec - \rvec \svec \right)
                + c_{5} \rvec \svec \rvec 
                + c_{6} \left(\svec^2 \rvec - \rvec \svec^2 \right)
                + c_{7} \left( \rvec \svec \rvec^2 - \rvec^2 \svec \rvec \right) 
                \label{eq:SGS-tf-8}
\end{equation}
we can examine the energy production contributions term-by-term in the 
eigen-frame where
\begin{equation}
    \svec =
    \begin{pmatrix}
        S_{1}      & 0         & 0         \\
        0           & S_{2}    & 0         \\
        0           & 0         & S_{3}    \\
    \end{pmatrix}
    \label{eq:Sij}
\end{equation}
and,
\begin{equation}
    \rvec =
    \begin{pmatrix}
        0           & R_{12}    & -R_{31}   \\
        -R_{12}     & 0         & R_{23}    \\
        R_{31}      & -R_{23}   & 0         \\
    \end{pmatrix}
\end{equation}


\subsubsection{$c_{1}$ Term}
It is easy to show that the first term in the SGS representation is always
stable by expanding and simplifying the complete term using
Eq.~\ref{eq:Sij}.
\begin{equation}
    \begin{split}
        \mathbf{S}:\mathbf{S} = S_{ij}S_{ij} = &
            S_{1}S_{1} + \cancel{S_{12} S_{12}} + \cancel{S_{13} S_{13}} +
            \cancel{S_{21}S_{21}} + S_{2} S_{2} + \cancel{S_{23} S_{23}} +
            \cancel{S_{31}S_{31}}                                   \\
        &   + \cancel{S_{32} S_{32}} + S_{3} S_{3}                   \\
    \end{split}
\end{equation}
Thus
%\begin{equation}
    \colorboxed{%
        \mathbf{S}:\mathbf{S} = S_{ij}S_{ij} =  S_{1}^{2} + S_{2}^{2} + S_{3}^{2} \geq 0 
    }
%\end{equation}

\subsubsection{$c_{2}$ Term}
Next we expand and simplify the $c_{2}$ term using Eq.~\ref{eq:Sij},
\begin{equation}
    \begin{split}
        \mathbf{S}:\mathbf{S}^{2} = S_{ij}(S_{ik} S_{kj}) = &
                S_{1}(S_{1} S_{1} + \cancel{S_{12} S_{21}}+ \cancel{S_{13} S_{31}}) + \\
            &   S_{2}(\cancel{S_{21} S_{12}} + S_{2} S_{2} + \cancel{S_{23} S_{32}}) + \\
            &   S_{3}(\cancel{S_{31} S_{13}} + \cancel{S_{32} S_{23}} + S_{3} S_{3})
    \end{split}
\end{equation}
This gives
\begin{equation}
    \mathbf{S}:\mathbf{S}^{2} = 
        S_{1}^3 + S_{2}^{3} + S_{3}^{3}
\end{equation}
Next we can substitute in the following expression for $S_{2}$
\begin{equation}
    S_{2} =  -(S_{1} + S_{3})
\end{equation}
and simplify to get the following
\begin{subequations}
    \begin{align}
        \mathbf{S}:\mathbf{S}^2 & =  
                S_{1}^3 - (S_{1} + S_{3})^{3} + S_{3}^{3}       \\
                & = \cancel{S_{1}^3} - (\cancel{S_{1}^3} + 3S_{1}^2 S_{3} + 3 S_{1} S_{3}^2 + \cancel{S_{3}^3}) + \cancel{S_{3}^3}              \\
                & = \underbrace{-S_{1}^2 S_{3}}_{\geq 0} \underbrace{- 3 S_{1} S_{3}^2}_{\leq 0}                                                
    \end{align}
\end{subequations}

Therefore $\mathbf{S}:\mathbf{S}^2 \leq 0$ if the following is true
\begin{subequations}
    \begin{equation}
        S_{1} S_{3}^2 \geq S_{1}^2 S_{3}
    \end{equation}
    %\begin{equation}
        \colorboxed{%
            \frac{S_{1}}{S_{3}} \leq 1
        }
    %\end{equation}
\end{subequations}

\newcommand{\term}{\mathbf{S}:\mathbf{R}^2}
\subsubsection{$c_{3}$ Term}
Continuing the process with the $c_{3}$ term and applying Eq.~\ref{eq:Sij}
we start by expanding and simplifying, namely,
\begin{equation}
    \begin{split}
        \mathbf{S}:\mathbf{R}^2 =  S_{ij}(R_{ik}R_{kj}) = &  
                S_{1}(\cancel{R_{11} R_{11}} + R_{12} R_{21} + R_{13} R_{31}) +     \\
            &   S_{2}(R_{21} R_{12} + \cancel{R_{22} R_{22}} + R_{23} R_{32}) +     \\
            &   S_{3}(R_{31} R_{13} + R_{32} R_{23} + \cancel{R_{33} R_{33}})
    \end{split}
\end{equation}
Substituting in $R_{21} = -R_{12}$, $R_{32} = -R_{23}$, and $R_{13} = -
R_{31}$ then gives
\begin{equation}
        \term = 
        S_{1} (-R_{12}^2 - R_{31}^2)  +
        S_{2} (-R_{12}^2 - R_{23}^2)  +
        S_{3} (-R_{31}^2 - R_{23}^2)
\end{equation}
This can be arranged as
\begin{equation}
        \term = 
        R_{12}^{2} (\underbrace{-S_{1} - S_{2}}_{= S_{3}}) + 
        R_{23}^{2} (\underbrace{-S_{2} - S_{3}}_{= S_{1}}) + 
        R_{31}^{2} (\underbrace{-S_{1} - S_{3}}_{= S_{2}})  
\end{equation}
Thus
\begin{subequations}
    \begin{align}
        \term = & 
            R_{12}^2 S_{3} + R_{23}^2 S_{1} + R_{31}^2 S_{2}                  \\
         = & R_{12}^2 S_{3} + R_{23}^2 S_{1} + R_{31}^2 (-S_{1} - S_{3})       \\
         = & R_{12}^2 S_{3} + R_{23}^2 S_{1} - R_{31}^2 S_{1} - R_{31}^2 S_{3} \\ 
         = & \underbrace{S_{1} R_{23}^2 - S_{3} R_{31}^{2}}_{\geq 0}          
                \underbrace{-S_{1} R_{31}^2 + S_{3} R_{12}^2}_{\leq 0}
    \end{align}
\end{subequations}

Therefore $\mathbf{S}:\mathbf{R}^2 \leq 0$ when the following is true
\begin{subequations}
    \begin{equation}
        S_{1} R_{23}^2 - S_{3} R_{31}^2 \leq S_{1} R_{31}^2 - S_{3} R_{12}^2
    \end{equation}
    \begin{equation}
        S_{1} (R_{23}^2 - R_{31}^{2}) \leq S_{3} ( R_{31}^2 - R_{12}^2)
    \end{equation}
    \label{eq:c3-inequal}
\end{subequations}     \\

Recall from the hand written course pack (pg.48) that the rotation rates
can be written in terms of the vorticity as,
\begin{equation}
    \mathbf{\omega} \equiv 2 (R_{23}, R_{31}, R_{12}) = (\omega_{1}, \omega_{2}, \omega_{3} )
\end{equation}
therefore,
\begin{subequations}
    \begin{equation}
        R_{23} = \frac{1}{2} \omega_{1}
    \end{equation}
    \begin{equation}
        R_{31} = \frac{1}{2} \omega_{2}
    \end{equation}
    \begin{equation}
        R_{12} = \frac{1}{2} \omega_{3}
    \end{equation}
    \label{eq:vorticity}
\end{subequations}

Substituting Eq.~\ref{eq:vorticity} into Eq.~\ref{eq:c3-inequal} gives
\begin{equation}
    S_{1} ( \omega_{1}^2 - \omega_{2}^2 ) \leq S_{3} (\omega_{2}^2 - \omega_{3}^2)
\end{equation}
Thus $\mathbf{S}:\mathbf{R}^2 \leq 0$ when
\begin{equation}
    \colorboxed{red}{
        \frac{ S_{3} ( \omega_{2}^2 - \omega_{3}^2 )}{S_{1} (\omega_{1}^2 - \omega_{2}^2 )} \geq 1
    }
\end{equation}

\subsubsection{$c_{4}$ Term}
Next using Eq.~\ref{eq:Sij} we will show that the $c_{4}$ term in the eigen-frame is always zero
by expanding the two terms and evaluating them separately, namely,
\begin{equation}
    \mathbf{S}:( \mathbf{SR} - \mathbf{RS}) = 
        \underbrace{S_{ij}(S_{kj}R_{kj})}_{\equiv \text{term 1}} 
        \underbrace{ - S_{ij}(R_{ik}S_{kj})}_{\equiv \text{term 2}}
\end{equation}
\subsubsection{Term 1}
Evaluating term 1 gives,
\begin{equation}
    \begin{split}
        S_{ij}(S_{ik}R_{kj}) = &
                S_{1}(S_{1} \cancel{R_{11}} + \cancel{S_{12}} R_{21} + \cancel{S_{13}} R_{31}) + \\
            &   S_{2}(\cancel{S_{21}} R_{12} + S_{2} \cancel{R_{22}} + \cancel{S_{23}} R_{32}) + \\
            &   S_{3}(\cancel{S_{31}} R_{13} + \cancel{S_{32}} R_{23} + S_{3} \cancel{R_{33}})
    \end{split}
\end{equation}
Therefore
\begin{equation}
    S_{ij}(S_{ik}R_{kj}) = 0 
\end{equation}
\subsubsection{Term 2}
Evaluating term 2 gives,
\begin{equation}
    \begin{split}
        S_{ij}(R_{ik}S_{kj}) = &
               -S_{1}(\cancel{R_{11}} S_{1} + R_{12} \cancel{S_{21}} + R_{13} \cancel{S_{31}})+   \\
            &  -S_{2}(R_{21} \cancel{S_{12}} + \cancel{R_{22}} S_{2} + R_{23} \cancel{S_{32}})+   \\
            &  -S_{3}(R_{31} \cancel{S_{13}} + R_{32} \cancel{S_{32}} + \cancel{R_{33}} S_{3})
    \end{split}
\end{equation}
Therefore
\begin{equation}
    S_{ij}(R_{ik}S_{kj}) = 0
\end{equation}
Thus
\begin{equation}
    \colorboxed{red}{
        \mathbf{S}:(\mathbf{SR} - \mathbf{RS}) =0 
    }
\end{equation}

\subsubsection{$c_{5}$ Term}
Next we Eq.~\ref{eq:Sij} to evaluate the $c_{5}$ term by expanding and
simplifying the following (note: terms in red are the only non-zero terms), 
%\begin{equation}
%    \begin{split}
%        \mathbf{S}:\mathbf{RSR} = S_{ij}(R_{ik}S_{kl} R_{lj}) = &
%                S_{1}(
%                \cancel{R_{11}}S_{1}\cancel{R_{11}}+ \cancel{R_{11}}\cancel{S_{12}}R_{21} + \cancel{R_{11}}\cancel{S_{13}}R_{31} +   \\
%            &   R_{12}\cancel{S_{21}}\cancel{R_{11}}+ {\color{red}R_{12}S_{2}R_{21}} + R_{12}\cancel{S_{23}}R_{31} +   \\
%            &   R_{13}\cancel{S_{31}}\cancel{R_{11}}+ R_{13}\cancel{S_{32}}R_{21} + {\color{red}R_{13}S_{3}R_{31}}) 
%            +    \\
%            &   S_{2}(
%                {\color{red}R_{21}S_{1}R_{12} }+ R_{21}\cancel{S_{12}}\cancel{R_{22}} + R_{21}\cancel{S_{13}}R_{32}) +  \\
%            &   \cancel{R_{22}}\cancel{S_{21}}R_{12}+ \cancel{R_{22}}S_{2}\cancel{R_{22}} + \cancel{R_{22}}\cancel{S_{23}}R_{32}) +  \\
%            &   R_{23}\cancel{S_{31}}R_{12}+ R_{23}\cancel{S_{32}}\cancel{R_{22}} + {\color{red}R_{23}S_{3}R_{32}})
%            +   \\
%            &   S_{3}(
%                {\color{red}R_{31}S_{1}R_{13}} + R_{31}\cancel{S_{12}}R_{23} + R_{31}\cancel{S_{13}}\cancel{R_{33}}) +  \\
%            &   R_{32}\cancel{S_{21}}R_{13}+ {\color{red}R_{32}S_{2}R_{23}} + R_{32}\cancel{S_{23}}\cancel{R_{33}}) +  \\
%            &   \cancel{R_{33}}\cancel{S_{31}}R_{13}+ \cancel{R_{33}}\cancel{S_{32}}R_{23} + \cancel{R_{33}}S_{3}\cancel{R_{33}})
%    \end{split}
%\end{equation}
%This simplifies to
%\begin{subequations}
%    \begin{equation}
%        \begin{split}
%            S_{ij}(R_{ik}S_{kl} R_{lj}) = &
%                S_{1}(R_{12}S_{2}R_{21} + R_{13} S_{3} R_{31})   +   \\
%            &   S_{2}(R_{21} S_{1} R_{12} + R_{23} S_{23} R_{32}) +   \\
%            &   S_{3}(R_{31}S_{1}R_{13} + R_{32} S_{2} R_{23})
%        \end{split}
%    \end{equation}
%    \begin{equation}
%        \begin{split}
%            S_{ij}(R_{ik}S_{kl} R_{lj}) = &
%                -S_{1} S_{2} R_{12} R_{12} - S_{1} S_{2} R_{12} R_{12} +    \\
%            &   -S_{1} S_{3} R_{31} R_{31} - S_{1} S_{3} R_{31} R_{31} +    \\
%            &    -S_{2} S_{3} R_{23} R_{23} - S_{2} S_{3} R_{31} R_{31} 
%        \end{split}
%    \end{equation}
%    \begin{equation}
%        S_{ij}(R_{ik}S_{kl} R_{lj}) = 
%            -2 S_{1} S_{2} R_{12}^2 -2 S_{1}S_{3} R_{31}^2 - 2 S_{2} S_{3} R_{23}^2
%    \end{equation}
%\end{subequations}
\begin{subequations}
\begin{align}
        \mathbf{S}:\mathbf{RSR} = S_{ij}(R_{ik}S_{kl} R_{lj}) = &
                S_{1}(
                \cancel{R_{11}}S_{1}\cancel{R_{11}}+ \cancel{R_{11}}\cancel{S_{12}}R_{21} + \cancel{R_{11}}\cancel{S_{13}}R_{31} +   \notag		\\
            &   R_{12}\cancel{S_{21}}\cancel{R_{11}}+ {\color{red}R_{12}S_{2}R_{21}} + R_{12}\cancel{S_{23}}R_{31} +   \notag		\\
            &   R_{13}\cancel{S_{31}}\cancel{R_{11}}+ R_{13}\cancel{S_{32}}R_{21} + {\color{red}R_{13}S_{3}R_{31}}) 
            +    \notag		\\
            &   S_{2}(
                {\color{red}R_{21}S_{1}R_{12} }+ R_{21}\cancel{S_{12}}\cancel{R_{22}} + R_{21}\cancel{S_{13}}R_{32}) +  		\\
            &   \cancel{R_{22}}\cancel{S_{21}}R_{12}+ \cancel{R_{22}}S_{2}\cancel{R_{22}} + \cancel{R_{22}}\cancel{S_{23}}R_{32}) +  \notag		\\
            &   R_{23}\cancel{S_{31}}R_{12}+ R_{23}\cancel{S_{32}}\cancel{R_{22}} + {\color{red}R_{23}S_{3}R_{32}}) +   \notag		\\
            &   S_{3}(
                {\color{red}R_{31}S_{1}R_{13}} + R_{31}\cancel{S_{12}}R_{23} + R_{31}\cancel{S_{13}}\cancel{R_{33}}) +  \notag		\\
            &   R_{32}\cancel{S_{21}}R_{13}+ {\color{red}R_{32}S_{2}R_{23}} + R_{32}\cancel{S_{23}}\cancel{R_{33}}) +  \notag		\\
            &   \cancel{R_{33}}\cancel{S_{31}}R_{13}+ \cancel{R_{33}}\cancel{S_{32}}R_{23} + \cancel{R_{33}}S_{3}\cancel{R_{33}})   \notag \\
\notag		\\
            = &
                S_{1}(R_{12}S_{2}R_{21} + R_{13} S_{3} R_{31})   +   \notag		\\
            &   S_{2}(R_{21} S_{1} R_{12} + R_{23} S_{23} R_{32}) +   		\\
            &   S_{3}(R_{31}S_{1}R_{13} + R_{32} S_{2} R_{23})      \notag  \\
\notag  \\
            = &
            -S_{1} S_{2} R_{12} R_{12} - S_{1} S_{2} R_{12} R_{12} +    \notag \\
            &   -S_{1} S_{3} R_{31} R_{31} - S_{1} S_{3} R_{31} R_{31} +    \\
            &    -S_{2} S_{3} R_{23} R_{23} - S_{2} S_{3} R_{31} R_{31} \notag \\    
\notag  \\ 
     = & 
    -2 S_{1} S_{2} R_{12}^2 -2 S_{1}S_{3} R_{31}^2 - 2 S_{2} S_{3} R_{23}^2
\end{align}
    \label{eq:c5-term}
\end{subequations}

Substituting in $S_{2} = -S_{1} -S_{3}$ into Eq.~\ref{eq:c5-term} we get the following
\begin{equation}
    \mathbf{S}:\mathbf{SRS} = 
        -2 S_{1} (-S_{1}-S_{3}) R_{12}^2 -2 S_{1}S_{3} R_{31}^2 - 2 (-S_{1}-S_{3}) S_{3} R_{23}^2
\end{equation}
Factoring the parenthesis and combing like terms then gives
\begin{equation}
    \mathbf{S}:\mathbf{SRS} = 
        2 S_{1}^2 R_{12}^2 +2 S_{1}S_{3} R_{12}^2 -2S_{1}S_{3}R_{31}^2 + 2 S_{1} S_{3} R_{23}^2 + 2 S_{3}^2 R_{23}^2
\end{equation}
Therefore $\mathbf{S}:\mathbf{RSR} \leq 0$ when
\begin{subequations}
    \begin{equation}
        2 S_{1}^2 R_{12}^2 +2 S_{1}S_{3} R_{12}^2 -2S_{1}S_{3}R_{31}^2 + 2 S_{1} S_{3} R_{23}^2 + 2 S_{3}^2 R_{23}^2 \leq 0
    \end{equation}
    \begin{equation}
        S_{1}^2 R_{12}^2 + 2 S_{3} R_{23}^2 \leq S_{1} S_{3} (R_{31}^2 - R_{12}^2 - R_{23}^2)
    \end{equation}
    \text{Substituting in Eq.~\ref{eq:vorticity} gives}
    \begin{equation}
        S_{1}^2 \omega_{3}^2 + S_{3}^{2} \omega_{1}^2 \leq S_{1} S_{3} (\omega_{2}^2 - \omega_{3}^2 - \omega_{1}^2)
    \end{equation}
\end{subequations}

Thus
%\begin{equation}
    \colorboxed{%
        \frac{\omega_{2}^2 - \omega_{3}^2 - \omega_{1}^2}
        {\left(\frac{S_{1}}{S_{3}}\right)\omega_{3}^2 - \left(\frac{S_{3}}{S_{1}}\right) \omega_{1}^2} \geq 1
    }
%\end{equation}


\subsubsection{$c_{6}$ Term}
Starting by separating the $c_{6}$ production term into two terms
and evaluating them individually gives the following,
\begin{equation}
    \mathbf{S}:(\mathbf{S}^2\mathbf{R} - \mathbf{RS}^2) =
        \underbrace{S_{ij}(S_{ik}S_{kl}R_{lk})}_{\equiv \text{term 1}}
        -\underbrace{S_{ij}(R_{ik}S_{kl}S_{lk})}_{\equiv \text{term 2}}
\end{equation}
%-------------------------------------------------------------------------%
% Term 1                                                                  %
%-------------------------------------------------------------------------%
\subsubsection{Term 1}
\begin{equation}
    \begin{split}
        S_{ij}(S_{ik}S_{kl}R_{lj}) = & 
            S_{1}(
            S_{1}S_{1}\cancel{R_{11}} + S_{1}\cancel{S_{12}}R_{21} + S_{1}\cancel{S_{13}}R_{31} +   \\
        &   \cancel{S_{12}}\cancel{S_{21}}\cancel{R_{11}} + \cancel{S_{12}}S_{2}R_{21} + \cancel{S_{12}}\cancel{S_{23}}R_{31} +   \\
        &   \cancel{S_{13}}\cancel{S_{31}}\cancel{R_{11}} + \cancel{S_{13}}\cancel{S_{32}}R_{21} + \cancel{S_{13}}S_{3}R_{31} 
            ) +   \\[6pt]
        &   S_{2}(
            \cancel{S_{21}}S_{1}R_{12} + \cancel{S_{21}}\cancel{S_{12}}\cancel{R_{22}} + \cancel{S_{21}}\cancel{S_{13}}R_{32} +   \\
        &   S_{2}\cancel{S_{21}}R_{12} + S_{2}S_{2}\cancel{R_{22}} + S_{2}\cancel{S_{23}}R_{32} +   \\
        &   \cancel{S_{23}}\cancel{S_{31}}R_{12} + \cancel{S_{23}}\cancel{S_{32}}\cancel{R_{22}} + \cancel{S_{23}}S_{3}R_{32} 
            ) +   \\[6pt]
        &   S_{3}(
            \cancel{S_{31}}S_{1}R_{13} + \cancel{S_{31}}\cancel{S_{12}}R_{23} + \cancel{S_{31}}\cancel{S_{13}}\cancel{R_{33}} +   \\
        &   \cancel{S_{32}}\cancel{S_{21}}R_{13} + \cancel{S_{32}}S_{2}R_{23} + \cancel{S_{32}}\cancel{S_{23}}\cancel{R_{33}} +   \\
        &   S_{3}\cancel{S_{31}}R_{13} + S_{3}\cancel{S_{32}}R_{23} + S_{3}S_{3}\cancel{R_{33}} )
    \end{split}
\end{equation}
where we can observe that there is no non-zero term, thus,
\begin{equation}
        S_{ij}(S_{ik}S_{kl}R_{lj}) = 0
\end{equation}
%-------------------------------------------------------------------------%
% Term 2                                                                  %
%-------------------------------------------------------------------------%
%\newpage
\subsubsection{Term 2}
\begin{equation}
    \begin{split}
        S_{ij}(R_{ik}S_{kl}S_{lj}) = & 
            S_{1}(
            \cancel{R_{11}}S_{1}S_{1} + \cancel{R_{11}}\cancel{S_{12}}\cancel{S_{21}} + \cancel{R_{11}}\cancel{S_{13}}\cancel{S_{31}}    +   \\
        &   R_{12}\cancel{S_{21}}S_{1} + R_{12}S_{2}\cancel{S_{21}} + R_{12}\cancel{S_{23}}\cancel{S_{31}}    +   \\
        &   R_{13}\cancel{S_{31}}S_{1} + R_{13}\cancel{S_{32}}\cancel{S_{21}} + R_{13}S_{3}\cancel{S_{31}}
            ) + \\[6pt]
        &   S_{2}(
            R_{21}S_{1}\cancel{S_{12}} + R_{21}\cancel{S_{12}}S_{2} + R_{21}\cancel{S_{13}}\cancel{S_{32}}    +   \\
        &   \cancel{R_{22}}\cancel{S_{21}}\cancel{S_{12}} + \cancel{R_{22}}S_{2}S_{2} + \cancel{R_{22}}\cancel{S_{23}}\cancel{S_{32}}    +   \\
        &   R_{23}\cancel{S_{31}}\cancel{S_{12}} + R_{23}\cancel{S_{32}}S_{2} + R_{23}S_{3}\cancel{S_{32}}
            ) + \\[6pt]
        &   S_{3}(
            R_{31}S_{1}\cancel{S_{13}} R_{31}\cancel{S_{12}}\cancel{S_{23}} R_{31}\cancel{S_{13}}S_{3}        +   \\
        &   R_{32}\cancel{S_{21}}\cancel{S_{13}} R_{32}S_{2}\cancel{S_{23}} R_{32}\cancel{S_{23}}S_{3}        +   \\
        &   \cancel{R_{33}}\cancel{S_{31}}\cancel{S_{13}} \cancel{R_{33}}\cancel{S_{32}}\cancel{S_{23}} \cancel{R_{33}}S_{3}S_{3})
    \end{split}
\end{equation}
where again it is obvious there are no no-zero terms, thus,
\begin{equation}
        S_{ij}(S_{ik}S_{kl}R_{lj}) = 0
\end{equation}
and more importantly
%\begin{equation}
    \colorboxed{
        \mathbf{S}:(\mathbf{S}^2\mathbf{R} - \mathbf{RS}^2) = 0
    }
%\end{equation}

%\newpage
\subsubsection{$c_7$ Term}
Again splitting the $c_{7}$ term into two terms and evaluating each terms
separately we get the following,
\begin{equation}
    \svec : (\rvec \svec \rvec^2 - \rvec^2 \svec \rvec) =
                \underbrace{ S_{ij} (R_{ik} S_{kl} R_{lm} R_{mj})}_{\equiv \text{term 1}} - 
                \underbrace{ S_{ij} (R_{ik} R_{kl} S_{lm} R_{mj})}_{\equiv \text{term 2}}
\end{equation}
\subsubsection{Term 1}
Expanding the first term gives,
\begin{equation}
    \begin{split}
        S_{ij}(R_{ik}S_{kl} R_{lm} R_{mj}) = &
            S_{1}(
            \cancel{R_{11}}S_{1}\cancel{R_{11}}\cancel{R_{11}} + \cancel{R_{11}}S_{1}R_{12}R_{21} + \cancel{R_{11}}S_{1}R_{13}R_{31}  +   \\
        &   \cancel{R_{11}}\cancel{S_{12}}R_{21}\cancel{R_{11}} + \cancel{R_{11}}\cancel{S_{12}}\cancel{R_{22}}R_{21} + \cancel{R_{11}}\cancel{S_{12}}R_{23}R_{31}  +   \\
        &   \cancel{R_{11}}S_{13}R_{31}\cancel{R_{11}} + \cancel{R_{11}}S_{13}R_{32}R_{21} + \cancel{R_{11}}S_{13}\cancel{R_{33}}R_{31}  +   \\
        &   R_{12}\cancel{S_{21}}\cancel{R_{11}}\cancel{R_{11}} + R_{12}\cancel{S_{21}}R_{12}R_{21} + R_{12}\cancel{S_{21}}R_{13}R_{31}  +   \\
        &   R_{12}S_{2}R_{21}\cancel{R_{11}} + R_{12}S_{2}\cancel{R_{22}}R_{21} + {\color{red}R_{12}S_{2}R_{23}R_{31}}  +   \\
        &   R_{12}\cancel{S_{23}}R_{31}\cancel{R_{11}} + R_{12}\cancel{S_{23}}R_{32}R_{21} + R_{12}\cancel{S_{23}}\cancel{R_{33}}R_{31}  +   \\
        &   R_{13}S_{31}\cancel{R_{11}}\cancel{R_{11}} + R_{13}S_{31}R_{12}R_{21} + R_{13}S_{31}R_{13}R_{31}  +   \\
        &   R_{13}\cancel{S_{32}}R_{21}\cancel{R_{11}} + R_{13}\cancel{S_{32}}\cancel{R_{22}}R_{21} + R_{13}\cancel{S_{32}}R_{23}R_{31}  +   \\
        &   R_{13}S_{3}R_{31}\cancel{R_{11}} + {\color{red}R_{13}S_{3}R_{32}R_{21}} + R_{13}S_{3}\cancel{R_{33}}R_{31} 
        )+ \\[6pt] 
        &   S_{2}(
            R_{21}S_{1}\cancel{R_{11}}R_{12} + R_{21}S_{1}R_{12}\cancel{R_{22}} + {\color{red}R_{21}S_{1}R_{13}R_{32}} +    \\
        &   R_{21}\cancel{S_{12}}R_{21}R_{12} + R_{21}\cancel{S_{12}}\cancel{R_{22}}\cancel{R_{22}} + R_{21}\cancel{S_{12}}R_{23}R_{32} +    \\
        &   R_{21}S_{13}R_{31}R_{12} + R_{21}S_{13}R_{32}\cancel{R_{22}} + R_{21}S_{13}\cancel{R_{33}}R_{32} +    \\
        &   \cancel{R_{22}}\cancel{S_{21}}\cancel{R_{11}}R_{12} + \cancel{R_{22}}\cancel{S_{21}}R_{12}\cancel{R_{22}} + \cancel{R_{22}}\cancel{S_{21}}R_{13}R_{32} +    \\
        &   \cancel{R_{22}}S_{2}R_{21}R_{12} + \cancel{R_{22}}S_{2}\cancel{R_{22}}\cancel{R_{22}} + \cancel{R_{22}}S_{2}R_{23}R_{32} +    \\
        &   \cancel{R_{22}}\cancel{S_{23}}R_{31}R_{12} + \cancel{R_{22}}\cancel{S_{23}}R_{32}\cancel{R_{22}} + \cancel{R_{22}}\cancel{S_{23}}\cancel{R_{33}}R_{32} +    \\
        &   R_{23}S_{31}\cancel{R_{11}}R_{12} + R_{23}S_{31}R_{12}\cancel{R_{22}} + R_{23}S_{31}R_{13}R_{32} +    \\
        &   R_{23}\cancel{S_{32}}R_{21}R_{12} + R_{23}\cancel{S_{32}}\cancel{R_{22}}\cancel{R_{22}} + R_{23}\cancel{S_{32}}R_{23}R_{32} +    \\
        &   {\color{red}R_{23}S_{3}R_{31}R_{12}} + R_{23}S_{3}R_{32}\cancel{R_{22}} + R_{23}S_{3}\cancel{R_{33}}R_{32} 
        ) + \\[6pt]
        &   S_{3}(
            R_{31}S_{1}\cancel{R_{11}}R_{13} + {\color{red}R_{31}S_{1}R_{12}R_{23}} + R_{31}S_{1}R_{13}\cancel{R_{33}} +    \\
        &   R_{31}\cancel{S_{12}}R_{21}R_{13} + R_{31}\cancel{S_{12}}\cancel{R_{22}}R_{23} + R_{31}\cancel{S_{12}}R_{23}\cancel{R_{33}} +    \\
        &   R_{31}S_{13}R_{31}R_{13} + R_{31}S_{13}R_{32}R_{23} + R_{31}S_{13}\cancel{R_{33}}\cancel{R_{33}} +    \\
        &   R_{32}\cancel{S_{21}}\cancel{R_{11}}R_{13} + R_{32}\cancel{S_{21}}R_{12}R_{23} + R_{32}\cancel{S_{21}}R_{13}\cancel{R_{33}} +    \\
        &   {\color{red}R_{32}S_{2}R_{21}R_{13}} + R_{32}S_{2}\cancel{R_{22}}R_{23} + R_{32}S_{2}R_{23}\cancel{R_{33}} +    \\
        &   R_{32}\cancel{S_{23}}R_{31}R_{13} + R_{32}\cancel{S_{23}}R_{32}R_{23} + R_{32}\cancel{S_{23}}\cancel{R_{33}}\cancel{R_{33}} +    \\
        &   \cancel{R_{33}}S_{31}\cancel{R_{11}}R_{13} + \cancel{R_{33}}S_{31}R_{12}R_{23} + \cancel{R_{33}}S_{31}R_{13}\cancel{R_{33}} +    \\
        &   \cancel{R_{33}}\cancel{S_{32}}R_{21}R_{13} + \cancel{R_{33}}\cancel{S_{32}}\cancel{R_{22}}R_{23} + \cancel{R_{33}}\cancel{S_{32}}R_{23}\cancel{R_{33}} +    \\
        &   \cancel{R_{33}}S_{3}R_{31}R_{13} + \cancel{R_{33}}S_{3}R_{32}R_{23} + \cancel{R_{33}}S_{3}\cancel{R_{33}}\cancel{R_{33}})
    \end{split}
\end{equation}
%\newpage
Thus
\begin{subequations}
    \begin{align}
        S_{ij}(R_{ik}S_{kl} R_{lm} R_{mj}) = &
                S_{1} ( R_{12}S_{2}R_{23}R_{31}       + R_{13} S_{3} R_{32} R_{21}) +           \notag  \\
            &   S_{2} ( R_{21} S_{1} R_{13} R_{32}    + R_{23} S_{3} R_{31} R_{12}) +                   \\
            &   S_{3} ( R_{31} S_{1} R_{12} R_{23}    + R_{32} S_{2} R_{21} R_{13})             \notag  \\
            \notag \\
            = &  
                S_{1} ( R_{12}S_{2}R_{23}R_{31}       - R_{31} S_{3} R_{23} R_{12}) +           \notag  \\
            &   S_{2} ( -R_{12} S_{1} R_{31} R_{23}   + R_{23} S_{3} R_{31} R_{12}) +                   \\
            &   S_{3} ( R_{31} S_{1} R_{12} R_{23}    - R_{23} S_{2} R_{12} R_{31})             \notag
    \end{align}
\end{subequations}

Factoring and grouping like terms clearly shows the first term is zero, namely,
\begin{equation}
    \begin{split}
        S_{ij}(R_{ik}S_{kl} R_{lm} R_{mj}) = &
            \underbrace{S_{1}S_{2}(R_{12}R_{23}R_{31} - R_{12}R_{23}R_{31})}_{=0} +   \\
        &   \underbrace{S_{1}S_{3}(R_{31}R_{12}R_{23} - R_{31}R_{12}R_{23})}_{=0} +   \\
        &   \underbrace{S_{2}S_{3}(R_{23}R_{31}R_{12} - R_{23}R_{31}R_{12})}_{=0}
    \end{split}
\end{equation}
Thus
\begin{equation}
    S_{ij}(R_{ik}S_{kl} R_{lm} R_{mj}) = 0
\end{equation}
%-------------------------------------------------------------------------%
% Term2                                                                   %
%-------------------------------------------------------------------------%
\subsubsection{Term 2}
\begin{equation}
    \begin{split}
        -S_{ij}(R_{ik}R_{kl} S_{lm} R_{mj}) = &
            -S_{1}(
                \cancel{R_{11}}\cancel{R_{11}}S_{1}\cancel{R_{11}} + \cancel{R_{11}}\cancel{R_{11}}\cancel{S_{12}}R_{21} + \cancel{R_{11}}\cancel{R_{11}}\cancel{S_{13}}R_{31} +    \\
            &   \cancel{R_{11}}R_{12}\cancel{S_{21}}\cancel{R_{11}} + \cancel{R_{11}}R_{12}S_{2}R_{21} + \cancel{R_{11}}R_{12}\cancel{S_{23}}R_{31} +    \\
            &   \cancel{R_{11}}R_{13}\cancel{S_{31}}\cancel{R_{11}} + \cancel{R_{11}}R_{13}\cancel{S_{32}}R_{21} + \cancel{R_{11}}R_{13}S_{3}R_{31} +    \\
            &   R_{12}R_{21}S_{1}\cancel{R_{11}} + R_{12}R_{21}\cancel{S_{12}}R_{21} + R_{12}R_{21}\cancel{S_{13}}R_{31} +    \\
            &   R_{12}\cancel{R_{22}}\cancel{S_{21}}\cancel{R_{11}} + R_{12}\cancel{R_{22}}S_{2}R_{21} + R_{12}\cancel{R_{22}}\cancel{S_{23}}R_{31} +    \\
            &   R_{12}R_{23}\cancel{S_{31}}\cancel{R_{11}} + R_{12}R_{23}\cancel{S_{32}}R_{21} + {\color{red}R_{12}R_{23}S_{3}R_{31}} +    \\
            &   R_{13}R_{31}S_{1}\cancel{R_{11}} + R_{13}R_{31}\cancel{S_{12}}R_{21} + R_{13}R_{31}\cancel{S_{13}}R_{31} +    \\
            &   R_{13}R_{32}\cancel{S_{21}}\cancel{R_{11}} + {\color{red}R_{13}R_{32}S_{2}R_{21}} + R_{13}R_{32}\cancel{S_{23}}R_{31} +    \\
            &   R_{13}\cancel{R_{33}}\cancel{S_{31}}\cancel{R_{11}} + R_{13}\cancel{R_{33}}\cancel{S_{32}}R_{21} + R_{13}\cancel{R_{33}}S_{3}R_{31}
            )     \\[6pt]
            &   -S_{2}(
                R_{21}\cancel{R_{11}}S_{1}R_{12} +R_{21}\cancel{R_{11}}\cancel{S_{12}}\cancel{R_{22}} +R_{21}\cancel{R_{11}}\cancel{S_{13}}R_{32} +      \\
            &   R_{21}R_{12}\cancel{S_{21}}R_{12} +R_{21}R_{12}S_{2}\cancel{R_{22}} +R_{21}R_{12}\cancel{S_{23}}R_{32} +      \\
            &   R_{21}R_{13}\cancel{S_{31}}R_{12} +R_{21}R_{13}\cancel{S_{32}}\cancel{R_{22}} +{\color{red}R_{21}R_{13}S_{3}R_{32}} +      \\
            &   \cancel{R_{22}}R_{21}S_{1}R_{12} +\cancel{R_{22}}R_{21}\cancel{S_{12}}\cancel{R_{22}} +\cancel{R_{22}}R_{21}\cancel{S_{13}}R_{32} +      \\
            &   \cancel{R_{22}}\cancel{R_{22}}\cancel{S_{21}}R_{12} +\cancel{R_{22}}\cancel{R_{22}}S_{2}\cancel{R_{22}} +\cancel{R_{22}}\cancel{R_{22}}\cancel{S_{23}}R_{32} +      \\
            &   \cancel{R_{22}}R_{23}\cancel{S_{31}}R_{12} +\cancel{R_{22}}R_{23}\cancel{S_{32}}\cancel{R_{22}} +\cancel{R_{22}}R_{23}S_{3}R_{32} +      \\
            &   {\color{red}R_{23}R_{31}S_{1}R_{12}} +R_{23}R_{31}\cancel{S_{12}}\cancel{R_{22}} +R_{23}R_{31}\cancel{S_{13}}R_{32} +      \\
            &   R_{23}R_{32}\cancel{S_{21}}R_{12} +R_{23}R_{32}S_{2}\cancel{R_{22}} +R_{23}R_{32}\cancel{S_{23}}R_{32} +      \\
            &   R_{23}\cancel{R_{33}}\cancel{S_{31}}R_{12} +R_{23}\cancel{R_{33}}\cancel{S_{32}}\cancel{R_{22}} +R_{23}\cancel{R_{33}}S_{3}R_{32}
            )       \\[6pt]
            &   -S_{3}(
                R_{31}\cancel{R_{11}}S_{1}R_{13} +R_{31}\cancel{R_{11}}\cancel{S_{12}}R_{23} +R_{31}\cancel{R_{11}}\cancel{S_{13}}\cancel{R_{33}}  +     \\
           &    R_{31}R_{12}\cancel{S_{21}}R_{13} +{\color{red}R_{31}R_{12}S_{2}R_{23}} +R_{31}R_{12}\cancel{S_{23}}\cancel{R_{33}}  +     \\
           &    R_{31}R_{13}\cancel{S_{31}}R_{13} +R_{31}R_{13}\cancel{S_{32}}R_{23} +R_{31}R_{13}S_{3}\cancel{R_{33}}  +     \\
           &    {\color{red}R_{32}R_{21}S_{1}R_{13}} +R_{32}R_{21}\cancel{S_{12}}R_{23} +R_{32}R_{21}\cancel{S_{13}}\cancel{R_{33}}  +     \\
           &    R_{32}\cancel{R_{22}}\cancel{S_{21}}R_{13} +R_{32}\cancel{R_{22}}S_{2}R_{23} +R_{32}\cancel{R_{22}}\cancel{S_{23}}\cancel{R_{33}}  +     \\
           &    R_{32}R_{23}\cancel{S_{31}}R_{13} +R_{32}R_{23}\cancel{S_{32}}R_{23} +R_{32}R_{23}S_{3}\cancel{R_{33}}  +     \\
           &    \cancel{R_{33}}R_{31}S_{1}R_{13} +\cancel{R_{33}}R_{31}\cancel{S_{12}}R_{23} +\cancel{R_{33}}R_{31}\cancel{S_{13}}\cancel{R_{33}}  +     \\
           &    \cancel{R_{33}}R_{32}\cancel{S_{21}}R_{13} +\cancel{R_{33}}R_{32}S_{2}R_{23} +\cancel{R_{33}}R_{32}\cancel{S_{23}}\cancel{R_{33}}  +     \\
           &    \cancel{R_{33}}\cancel{R_{33}}\cancel{S_{31}}R_{13} +\cancel{R_{33}}\cancel{R_{33}}\cancel{S_{32}}R_{23} +\cancel{R_{33}}\cancel{R_{33}}S_{3}\cancel{R_{33}})
    \end{split}
\end{equation}
%\newpage
Thus
\begin{subequations}
    \begin{align}
            -S_{ij}(R_{ik}R_{kl} S_{lm} R_{mj})   = &
                    -S_{1}  (R_{12} R_{23} S_{3} R_{31}   + R_{13} R_{32} S_{2} R_{21})      \notag \\
                &   -S_{2}  (R_{21} R_{13} S_{3} R_{32}   + R_{23} R_{31} S_{1} R_{12})             \\  
                &   -S_{3}  (R_{31} R_{12} S_{2} R_{23}   + R_{32} R_{21} S_{1} R_{13})      \notag \\  
                \notag \\
                = &
                    -S_{1}  (R_{12} R_{23} S_{3} R_{31}   - R_{31} R_{23} S_{2} R_{12})      \notag \\
                &   -S_{2}  (-R_{12} R_{31} S_{3} R_{23}  + R_{23} R_{31} S_{1} R_{12})             \\  
                &   -S_{3}  (R_{31} R_{12} S_{2} R_{23}   - R_{23} R_{12} S_{1} R_{31})     \notag   
    \end{align}
\end{subequations}

Factoring and grouping like terms clearly shows the first term is zero, namely,
    \begin{equation}
        \begin{split}
         -S_{ij}(R_{ik}R_{kl} S_{lm} R_{mj})   = &
                \underbrace{-S_{1}S_{2}(R_{31}R_{23}R_{12} - R_{31}R_{23}R_{12})}_{=0}    \\
            &   \underbrace{-S_{1}S_{3}(R_{23}R_{12}R_{31} - R_{23}R_{12}R_{31})}_{=0}    \\
            &   \underbrace{-S_{2}S_{3}(R_{12}R_{31}R_{23} - R_{12}R_{31}R_{12})}_{=0}
        \end{split}
    \end{equation}
Therefore
\begin{equation}
    -S_{ij}(R_{ik}R_{kl} S_{lm} R_{mj}) = 0
\end{equation}
which means that the entire $c_{7} $ term is zero, namely,
%\begin{equation}
    \colorboxed{%
        \svec : (\rvec \svec \rvec^2 - \rvec^2 \svec \rvec) = 0
    }
%\end{equation}

