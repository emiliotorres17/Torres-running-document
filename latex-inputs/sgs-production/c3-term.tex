\newcommand{\term}{\mathbf{S}:\mathbf{R}^2}
\subsubsection{$c_{3}$ Term}
Continuing the process with the $c_{3}$ term and applying Eq.~\ref{eq:Sij}
we start by expanding and simplifying, namely,
\begin{equation}
    \begin{split}
        \mathbf{S}:\mathbf{R}^2 =  S_{ij}(R_{ik}R_{kj}) = &  
                S_{1}(\cancel{R_{11} R_{11}} + R_{12} R_{21} + R_{13} R_{31}) +     \\
            &   S_{2}(R_{21} R_{12} + \cancel{R_{22} R_{22}} + R_{23} R_{32}) +     \\
            &   S_{3}(R_{31} R_{13} + R_{32} R_{23} + \cancel{R_{33} R_{33}})
    \end{split}
\end{equation}
Substituting in $R_{21} = -R_{12}$, $R_{32} = -R_{23}$, and $R_{13} = -
R_{31}$ then gives
\begin{equation}
        \term = 
        S_{1} (-R_{12}^2 - R_{31}^2)  +
        S_{2} (-R_{12}^2 - R_{23}^2)  +
        S_{3} (-R_{31}^2 - R_{23}^2)
\end{equation}
This can be arranged as
\begin{equation}
        \term = 
        R_{12}^{2} (\underbrace{-S_{1} - S_{2}}_{= S_{3}}) + 
        R_{23}^{2} (\underbrace{-S_{2} - S_{3}}_{= S_{1}}) + 
        R_{31}^{2} (\underbrace{-S_{1} - S_{3}}_{= S_{2}})  
\end{equation}
Thus
\begin{subequations}
    \begin{align}
        \term = & 
            R_{12}^2 S_{3} + R_{23}^2 S_{1} + R_{31}^2 S_{2}                  \\
         = & R_{12}^2 S_{3} + R_{23}^2 S_{1} + R_{31}^2 (-S_{1} - S_{3})       \\
         = & R_{12}^2 S_{3} + R_{23}^2 S_{1} - R_{31}^2 S_{1} - R_{31}^2 S_{3} \\ 
         = & \underbrace{S_{1} R_{23}^2 - S_{3} R_{31}^{2}}_{\geq 0}          
                \underbrace{-S_{1} R_{31}^2 + S_{3} R_{12}^2}_{\leq 0}
    \end{align}
\end{subequations}

Therefore $\mathbf{S}:\mathbf{R}^2 \leq 0$ when the following is true
\begin{subequations}
    \begin{equation}
        S_{1} R_{23}^2 - S_{3} R_{31}^2 \leq S_{1} R_{31}^2 - S_{3} R_{12}^2
    \end{equation}
    \begin{equation}
        S_{1} (R_{23}^2 - R_{31}^{2}) \leq S_{3} ( R_{31}^2 - R_{12}^2)
    \end{equation}
    \label{eq:c3-inequal}
\end{subequations}     \\

Recall from the hand written course pack (pg.48) that the rotation rates
can be written in terms of the vorticity as,
\begin{equation}
    \mathbf{\omega} \equiv 2 (R_{23}, R_{31}, R_{12}) = (\omega_{1}, \omega_{2}, \omega_{3} )
\end{equation}
therefore,
\begin{subequations}
    \begin{equation}
        R_{23} = \frac{1}{2} \omega_{1}
    \end{equation}
    \begin{equation}
        R_{31} = \frac{1}{2} \omega_{2}
    \end{equation}
    \begin{equation}
        R_{12} = \frac{1}{2} \omega_{3}
    \end{equation}
    \label{eq:vorticity}
\end{subequations}

Substituting Eq.~\ref{eq:vorticity} into Eq.~\ref{eq:c3-inequal} gives
\begin{equation}
    S_{1} ( \omega_{1}^2 - \omega_{2}^2 ) \leq S_{3} (\omega_{2}^2 - \omega_{3}^2)
\end{equation}
Thus $\mathbf{S}:\mathbf{R}^2 \leq 0$ when
%\begin{equation}
    \colorboxed{%
        \frac{ S_{3} ( \omega_{2}^2 - \omega_{3}^2 )}{S_{1} (\omega_{1}^2 - \omega_{2}^2 )} \geq 1
    }
%\end{equation}
