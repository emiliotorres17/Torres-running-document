\subsubsection{$c_{4}$ Term}
Next using Eq.~\ref{eq:Sij} we will show that the $c_{4}$ term in the eigen-frame is always zero
by expanding the two terms and evaluating them separately, namely,
\begin{equation}
    \mathbf{S}:( \mathbf{SR} - \mathbf{RS}) = 
        \underbrace{S_{ij}(S_{kj}R_{kj})}_{\equiv \text{term 1}} 
        \underbrace{ - S_{ij}(R_{ik}S_{kj})}_{\equiv \text{term 2}}
\end{equation}
\subsubsection{Term 1}
Evaluating term 1 gives,
\begin{equation}
    \begin{split}
        S_{ij}(S_{ik}R_{kj}) = &
                S_{1}(S_{1} \cancel{R_{11}} + \cancel{S_{12}} R_{21} + \cancel{S_{13}} R_{31}) + \\
            &   S_{2}(\cancel{S_{21}} R_{12} + S_{2} \cancel{R_{22}} + \cancel{S_{23}} R_{32}) + \\
            &   S_{3}(\cancel{S_{31}} R_{13} + \cancel{S_{32}} R_{23} + S_{3} \cancel{R_{33}})
    \end{split}
\end{equation}
Therefore
\begin{equation}
    S_{ij}(S_{ik}R_{kj}) = 0 
\end{equation}
\subsubsection{Term 2}
Evaluating term 2 gives,
\begin{equation}
    \begin{split}
        S_{ij}(R_{ik}S_{kj}) = &
               -S_{1}(\cancel{R_{11}} S_{1} + R_{12} \cancel{S_{21}} + R_{13} \cancel{S_{31}})+   \\
            &  -S_{2}(R_{21} \cancel{S_{12}} + \cancel{R_{22}} S_{2} + R_{23} \cancel{S_{32}})+   \\
            &  -S_{3}(R_{31} \cancel{S_{13}} + R_{32} \cancel{S_{32}} + \cancel{R_{33}} S_{3})
    \end{split}
\end{equation}
Therefore
\begin{equation}
    S_{ij}(R_{ik}S_{kj}) = 0
\end{equation}
Thus
\begin{equation}
    \colorboxed{red}{
        \mathbf{S}:(\mathbf{SR} - \mathbf{RS}) =0 
    }
\end{equation}
