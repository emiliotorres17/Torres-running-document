\newcommand{\bfvec}[1]{%
    \mathbf{#1}
    }
\newcommand{\filtuvec}{%
    \mathbf{\tilde{u}}
    }
\newcommand{\filtsvec}{%
    \mathbf{\tilde{S}}
    }
\newcommand{\filtwvec}{%
    \mathbf{\tilde{\omega}}
    }
\newcommand{\filterns}{%
    \pdv{\filtuvec}{t} + \mathbf{\tilde{u}} \cdot \grad{\mathbf{\tilde{u}}} = 
        -\grad{p} + \nu \laplacian{\mathbf{\tilde{u}}} - \div \mathbf{\tau}
    }
\section{Filtered Enstrophy Equation}
To derive the filtered enstrophy equation we start by determining the
filtered vorticity equation from the filtered Navier-Stokes equations,
\begin{subequations}
    \begin{equation}
        \div \filtuvec = 0
        \label{eq:filtered-mass}
    \end{equation}
    \begin{equation}
        \filterns
        \label{eq:filtered-momentum}
    \end{equation}
\end{subequations}

The transport equation for vorticity, $\mathbf{\tilde{\omega}} \equiv \curl
\filtuvec $, can be obtained by taking the curl of the above equation. 
Applying the curl to Eq.~\ref{eq:filtered-momentum} gives
\begin{subequations}
    \begin{equation}
        \curl \left(\filterns\right)
    \end{equation}
    \begin{equation}
        \pdv{\filtwvec}{t} + \filtuvec \cdot \grad \filtwvec =
            \filtwvec \cdot \grad \filtuvec + \nu \laplacian{\filtwvec}
            - \curl \div \mathbf{\tau}
    \end{equation}
    \label{eq:filtered-vorticity}
\end{subequations}

Note in deriving Eq.~\ref{eq:filtered-vorticity} we have applied the following vector
identities
\begin{subequations}
    \begin{equation}
        \bfvec{u} \cdot \grad \bfvec{u} = 
                \grad\left(\frac{1}{2}\bfvec{u} \cdot \bfvec{u}\right) 
                - \bfvec{u} \cp \left(\curl \bfvec{u}\right)
    \end{equation}
    \begin{equation}
        \curl \grad \phi = 0
    \end{equation}
    \begin{equation}
        \div \curl \mathbf{a} = 0
    \end{equation}
\end{subequations}

Next we can multiply Eq.~\ref{eq:filtered-vorticity} to obtain the filtered
enstrophy, $\Omega \equiv \frac{1}{2} \filtwvec \filtwvec$, transport
equation, namely,
\begin{subequations}
    \begin{equation}
        \filtwvec \cdot \bigg(
        \pdv{\filtwvec}{t} + \filtuvec \cdot \grad \filtwvec =
            \filtwvec \cdot \grad \filtuvec + \nu \laplacian{\filtwvec}
            - \curl \div \mathbf{\tau}
            \bigg)
    \end{equation}
    \begin{equation}
        \pdv{\Omega}{t} + \filtuvec \cdot \grad \Omega = 
            \filtwvec \cdot \filtsvec \cdot \filtwvec + \nu \laplacian{\Omega}
            - \nu \grad \filtwvec : \grad \filtwvec 
            - \filtwvec \cdot \curl \div \mathbf{\tau}
    \end{equation}
\end{subequations}


\subsection{Subgrid Stress Enstrophy Evaluation}
As we shall see, it is helpful to evaluate transport equations in terms of
their physical representation (e.g., transport, dissipation, and
production). In the following we will separate the subgrid stress term into
both transport and production terms in order to obtain physical insight.
We begin the evaluation by expressing the subgrid stress in index notation,
\begin{equation}
    -\filtwvec \cdot \curl \div \tau =
        - \varepsilon_{ijk} \tilde{\omega}_{i} 
        \pdv{}{x_{j}} \pdv{}{x_{l}} \tau_{kl}
\end{equation}
Interchanging the linear derivative operators and moving the Levi-Civita
inside the first derivative gives
\begin{equation}
        - \varepsilon_{ijk} \tilde{\omega}_{i}  \pdv{}{x_{j}} \pdv{}{x_{l}}
        \tau_{kl} =
        - \tilde{\omega}_{i}  \pdv{}{x_{l}} \underbrace{\varepsilon_{ijk} \pdv{}{x_{j}}
        \tau_{kl}}_{\equiv \curl \mathbf{\tau}}
\end{equation}
Substituting in $\Psi_{il}$ for $\curl \mathbf{\tau}$ the following can be 
arranged to get 
\begin{equation}
    \begin{split}
        - \tilde{\omega}_{i}  \pdv{}{x_{l}} \varepsilon_{ijk} \pdv{}{x_{j}} \tau_{kl}  = &
            -\tilde{\omega}_{i} \pdv{\Psi_{il}}{x_{l}} \\
            & = - \underbrace{\pdv{}{x_{l}}\big(\tilde{\omega}_{i} \Psi_{il} \big)}_{\equiv \text{Transport}}
            + \underbrace{\Psi_{il} \pdv{\tilde{\omega}_{i}}{x_{l}}}_{\equiv \text{Production}}
    \end{split}
    \label{eq:trans-enstrophy}
\end{equation}
Therefore from Eq.~\ref{eq:trans-enstrophy} it is clear that that the transport
of enstrophy from the subgrid stress is due to 
\begin{equation}
    \colorboxed{red}{
        - \pdv{}{x_{l}}\big(\tilde{\omega}_{i} \Psi_{il} \big)
    }
\end{equation}
while the production is caused by
\begin{equation}
    \colorboxed{red}{
        \Psi_{il} \pdv{\tilde{\omega}_{i}}{x_{l}}
    }
\end{equation}
